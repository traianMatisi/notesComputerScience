\documentclass[12pt, a4paper]{article}
\usepackage[utf8x]{inputenc} %acentuação
\usepackage[lmargin=2cm, tmargin=1cm, rmargin=2cm, bmargin=1cm]{geometry}
\usepackage{ucs}
\usepackage{textcomp}
\usepackage[T1]{fontenc} %ajusta textos copiados/colados
\usepackage[brazil]{babel}
\usepackage{amsmath, amsthm, amsfonts, amssymb, dsfont, mathtools} %símbolos matemáticos
\usepackage{float}
\usepackage{graphicx} %permite inserir figuras
\usepackage{tabto} %permitir tabulação
\usepackage{blindtext} %lorem ipsum se chamar a função \blindtext
%\input{file.tex}
\pagestyle{empty} %turn off page numbers (currently not working in my TeX distro)
\parindent 0px %turn off indentation

\begin{document}

\title{Introdução ao \LaTeX}
\author{Traian Matisi}
%\date{13 mar 2023}
%\headcm
%\todayr
\maketitle

Useful links
\begin{itemize}
    \item[i]Personally, I recomend to download the \LaTeX\ distro miktex for linux and/or windows at:\\
    miktex.org/download
    \item[ii]Also, I personally recommend as the text editor the texmaker (I use VSCode with some \LaTeX\ extension) you can download texmaker here:\\
    www.xm1math.net/texmaker/download.html
    \item[iii]Make ABNT format over here:\\
    www.abntex.net.br/
    \item[iv]To learn without modifying your computer you can use Overleaf online:\\
    https://www.overleaf.com/
    \item[v]There are other distros as well, my point is, you should use whatever you feel like and like. A'ight, after installation and launch, let's get started.
\end{itemize}

\section{Introduction}
Commands are written with backslash in front of words\\
This document header is writen as it follows (I used special characters for all)

\begin{itemize}
    \item[1]\textbackslash documentclass$[$12pt, a4paper$]$\{article\}
    \item[2]\textbackslash usepackage$[$utf8x$]$\{inputenc\}
    \item[3]\textbackslash usepackage$[$lmargin=2cm, tmargin=2cm, rmargin=2cm, bmargin=2cm$]$\{geometry\}
    \item[4]\textbackslash usepackage\{ucs\}
    \item[5]\textbackslash usepackage\{textcomp\}
    \item[6]\textbackslash usepackage$[$T1$]$\{fontenc\}
    \item[7]\textbackslash usepackage$[$brazil$]$\{babel\}
    \item[8]\textbackslash usepackage\{amsmath, amsthm, amsfonts, amssymb, dsfont, mathtools\}
    \item[9]\textbackslash usepackage\{float\}
    \item[10]\textbackslash usepackage\{graphicx\}
    \item[11]\textbackslash usepackage\{tabto\}
    \item[12]\textbackslash pagestyle\{empty\}
    \item[13]\textbackslash parindent 0px
    \item[14]\textbackslash title\{Introdução ao \textbackslash LaTeX \textbackslash\}
    \item[15]\textbackslash author\{Traian Matisi\}
    \item[16]\%\textbackslash date\{13 mar 2023\} commented to use compilation date
    \item[17]\textbackslash begin\{document\}
    \item[18]\ldots
    \item[19]\textbackslash end\{document\}
\end{itemize}

To compile the special characters we did it like this
\begin{itemize}
    \item[]\textbackslash \& $\rightarrow$ \&
    \item[]\textbackslash \% $\rightarrow$ \%
    \item[]\textbackslash \$ $\rightarrow$ \$
    \item[]\textbackslash \# $\rightarrow$ \#
    \item[]\textbackslash \_ $\rightarrow$ \_
    \item[]\textbackslash \{ $\rightarrow$ \{
    \item[]\textbackslash \} $\rightarrow$ \}
    \item[]\$$[$\$ $\rightarrow$ $[$
    \item[]\$$]$\$ $\rightarrow$ $]$
    \item[]\textbackslash textasciitilde $\rightarrow$ \textasciitilde
    \item[]\textbackslash textasciicircum $\rightarrow$ \textasciicircum
    \item[]\textbackslash textbackslash $\rightarrow$ \textbackslash
\end{itemize}

Return characters:\\
To make to the next line
\begin{itemize}
    \item[a]Hard return is a blank line, use for new paragraph
    \item[b]Soft return is \textbackslash\textbackslash, use for new line in paragraph
    \item[c]We can put a size argument after the double backslash \textbackslash\textbackslash$[$6cm$]$ or pt, in, mm, etc\ldots
\end{itemize}

\section{Formating}
%\begin{document}
%\end{document}
%
%\section{<name>}
%\subsection{<name>}
%
%\begin{center}
%\end{center}
%\begin{flushright}
%\end{flushright}
%\begin{flushleft}
%\end{flushleft}
%\begin{???justified???}
%\end{???justified???}
%
%\begin{itemize}
%\end{itemize}
%\begin{enumerate}
%\end{enumerate}
%
%\textbf{negrito/bold}
%\textit{italico/italico}
%\underline{sublinhado}
%
%ACIMA PODEM ANINHAR
%
%\begin{figure}[ht]
%\centering
%\includegraphics[width=2cm]{nome.png}
%\caption{Legenda dessa figura}
%\label{Imagem-01}
%\end{figure}
%
%\ref{Imagem-01}
%
%\vspace{5cm}
%
%\begin{array}{ccc...cc}
%   a & b & c & ... & y & z\\
%   A & B & C & ... & Y & Z\\
%\end{array}
%
%   $
%   \left[\begin{array}{ccc...cc}
%
%   a & b & c & ... & y & z\\
%   A & B & C & ... & Y & Z\\
%
%   \end{array}\right]
%   $
%
%\begin{equation}
%   $x=y$
%   \lim_{x\displaystyle\rightarrow 0}f(x)
%   \iiinto_0^1f(x)dx
%\end{equation}
%
%\begin{tabular}{!}{c|l|r|...|c|}\hline
%   \hline
%\end{tabular}
%
%\begin{table}{!}{c|l|r|...|c|}\hline
%   \caption{legenda}
%   \label{etiqueta}
%
%   \hline
%\end{table}
%
%\begin{figure}
%   \begin{minipage}[!]{0.35\linewidth}
%       \centering
%       \caption
%       \includegraphics[\width=\linewidth]{marca}
%      \label{my_label}
%   \end{minipage}
%   \begin{minipage}[!]{0.35\linewidth}
%       \centering
%       \caption
%       \includegraphics[\width=\linewidth]{marca}
%      \label{my_label}
%   \end{minipage}
%   \begin{minipage}[!]{0.35\linewidth}
%       \centering
%       \caption
%       \includegraphics[\width=\linewidth]{marca}
%      \label{my_label}
%   \end{minipage}
%   \begin{minipage}[!]{0.35\linewidth}
%       \centering
%       \caption
%       \includegraphics[\width=\linewidth]{marca}
%      \label{my_label}
%   \end{minipage}
%\end{figure}
%
%
%
%
%
%
%

\section{Math}
Common mathematical notations

\begin{itemize} %we can use \begin{enumerate}...\end{enumerate}
    \item[i] Sum: $x+y$
    \item[ii] Subtraction: $x-y$
    \item[iii] Product: $x\cdot y$
    \item[iii] Product: $x\times y$
    \item[iv] Division: $\frac{x}{y}$
    \item[iv] Division: $\dfrac{x}{y}$
    \item[iv] Division: $x\div y$
    \item[iv] Division: $x / y$
    \item[iv] Division: $x : y$
    \item[v] Sometimes the fractions are resized inside phrases like this $\displaystyle \frac{1}{2}$; $\frac{1}{2}$; $\dfrac{1}{2}$
\end{itemize}

Math symbols and letters needs the money sign as delimiters \$$2x^2$\$

Two dollar signs in each end makes the equation centered and in his own line\\
If we type \$\$2x\textasciicircum 2\$\$ we get: $$2x^2$$

We can have some problem with two digits exponentials, we should wrap the whole exponent (wich we call it superscript) with curly braces, if we want $x^{12}$ we should write it like this: $$\$x\textasciicircum\{12\}\$ \rightarrow x^{12}$$
The same is true for subscripts, for invoking such we use underline. To write $C_{6}H_{12}O_{6}$ we should write: $$\$C\_\{6\}H\_\{12\}O\_\{6\}\$ \rightarrow C_{6}H_{12}O_{6}$$

Subscripts can be nested, as so superscript.

Some greek letters are just words after the backslash

\begin{itemize}
    \item [] \textbackslash pi = $\pi$
    \item [] \textbackslash Pi = $\Pi$
    \item [] \textbackslash alpha = \textbackslash pi r\textasciicircum2 $\rightarrow$ $\alpha = \pi r^{2}$
\end{itemize}

Some trigonometrics functions

\begin{itemize}
    \item [] \$\textbackslash sin x\$ $\rightarrow \sin x$
    \item [] \$\textbackslash tan \textbackslash alpha\$ $\rightarrow \tan \alpha$
    \item [] \$\textbackslash cos\textasciicircum\{-1\} y\$ $\rightarrow \cos^{-1} y$
\end{itemize}

Logarithmics functions

\begin{itemize}
    \item [] \$\textbackslash log x\$ $\rightarrow \log x$
    \item [] \$y = \textbackslash log \_\{5\}x\$ $\rightarrow y = \log_{5} x$
\end{itemize}

Roots

\begin{itemize}
    \item [] \$\textbackslash sqrt\{16\}\$ $\rightarrow \sqrt{16}$
    \item [] \$\textbackslash sqrt$[$3$]$\{8\}\$ $\rightarrow \sqrt[3]{8}$
    \item [] With nested roots, we put the inside roots inside the curly brackets
\end{itemize}

Math notations

\begin{itemize}
    \item [] Parenthesis and set notation:\\$a(b+c) = ab + ac \rightarrow a, b, c \in \mathbb{R}$
    \item [] Parenthesis unscaled to equation size\\$2\left(\frac{1}{x^{2}-1}\right)$
    \item [] Parenthesis scaled to equation size\\$2\left(\dfrac{1}{x^{2}-1}\right)$
    \item [] Square brackets scaled to equation size\\$2\left[\dfrac{1}{x^{2}-1}\right]$
    \item [] Curly brackets scaled to equation size\\$2\left\{\dfrac{1}{x^{2}-1}\right\}$
    \item [] One side curly bracket scaled to equation size\\$\left\{\dfrac{x+y = 2}{x-y=-1}\right.$
    \item[] One side pipe scaled to equation size\\$\left.\dfrac{dy}{dx}\right|_{x=1}$
    \item[] Module brackets scaled to equation size\\$2\left|\dfrac{1}{x^{2}-1}\right|$
    \item[] Norm brackets scaled to equation size\\$2\left|\left|\dfrac{1}{x^{2}-1}\right|\right|$
    \item[] Angle brackets scaled to equation size\\$2\left\langle\dfrac{1}{x^{2}-1}\right\rangle$
    \item[] Parenthesis with inside parenthesis unscaled\\$\left(\frac{1}{1+\left(\frac{1}{1+x}\right)}\right)$
    \item[] Parenthesis with inside parenthesis scaled\\$\left(\dfrac{1}{1+\left(\dfrac{1}{1+x}\right)}\right)$
\end{itemize}

\section{Tables}

\begin{tabular}{|c|c|c|c|c|c|c|}\hline%c means centered, pipe makes a wall, r align in right, l in left
$x$ & 1 & 2 & 3 & 4 & 5 & 6 \\\hline
$f(x)$ & 10 & 11 & 12 & 13 & 14 & 15 \\\hline
\end{tabular}
%\vspace[1cm]
\end{document}