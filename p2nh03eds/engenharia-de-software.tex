\documentclass{article}

\begin{document}

-=-=-=-=-=-=-=-=-=-=-=-\\
| Engenharia de Software   |\\
-=-=-=-=-=-=-=-=-=-=-=-\\


Trabalho prático 4:

	-Realizar a inspeção do Documento de Requisitos E da Descrição dos Casos de Uso.
		Levar em consideração os critérios de qualidade dos requisitos e casos de uso.
		Levar em consideração a necessidade dos documentos estarem consistentes entre si.
	-A inspeção deve ser realizada a partir de duas atividades.
		--> Planejamento --> Detecção.
		planejamento: definir um checklist com questões para apoiar a inspeção.
		detecção: identificar defeitos, preenchendo o checklist, apresentando os problemas identificados e sugerindo alterações para corrigi-los.

-=-=-=-=-=-=-=-=-=-=-=-=-\\
| 18 - Gerência de qualidade   |\\
-=-=-=-=-=-=-=-=-=-=-=-=-\\

Qualidade é um atributo subjetivo. Contudo podemos escolher algumas abordagens para padronizar um controle de qualidade.
Primeiro, temos que dividir o processo de melhoria da qualidade em duas etapas.

	-Análise do problema.
	-Projeto da solução.

Os atributos de um software de qualidade variam de pessoa pra pessoa, mas podemos escolher algumas abordagens comuns.
Essas abordagens levam em conta algumas características do software:

	-Tamanho do software.
	-Complexidade do software.
	-Métodos, técnicas e ferramentas utilizadas.
	-Custo/benefício.
	-Custos associados à existência de erros.
	-Custos associados à detecção e remoção de erros.
	-Etcetera

Na história, tivemos 3 grandes "escolas de qualidade nas industrias":

	-Inspeção de produtos individualmente (industria artesanal).
	-Inspeção por amostragem (linha de produção).
	-Inspeção do processo produtivo.

Abordagens de qualidade:

	Abordagem Deming:

		-Qualidade inicia com alto nível gerencial.
		-Qualidade desde os fornecedores até o cliente final.
		-Fazer certo na primeira vez.
		-14 princípios.

	Abordagem Feigenbaum:

	Abordagem Crosby:

	Abordagem do Ciclo PDCA (plan-do-check-act) - Forma de controle de processo:

		-P
		-D
		-C
		-A

Ferramentas de qualidade:

	-Fluxograma.
	-Diagrama de Ishikawa.
	-Gráfico de Paretto
	-Histograma.
	-Diagrama de dispersão.
	-Folha de verificação.
	-Gráfico de controle.

Qualidade do produto versus Qualidade do processo.

	Qualidade do PRODUTO depende de:

		-Qualidade do PROCESSO.
		-Qualidade da equipe
		-Tecnologias do desenvolvimento.
		-Custo/Tempo/Cronograma

	Qualidade do PROCESSO influencia:

		-Atributos da qualidade interna do produto.
		-Atributos da qualidade externa do produto.
		-Atributos da qualidade de uso do produto.

		Para controlar a melhoria do processo:

			-Medir/avaliar o processo.
			-controlar o processo.
			-Executar o processo.
			-Melhorar o processo.
			-Definir o processo.

		Normas e modelos de maturidade da qualidade de processo:

			ISO/IEC 12.207
			ISO/IEC 33.0xx
			CMMI-DEV
			MR-MPS

-=-=-=-=-=-=-=-=-=-=-=-=-=-=-=-=-=-\\
| 20 - Planejamento e projetos de Testes   |\\
-=-=-=-=-=-=-=-=-=-=-=-=-=-=-=-=-=-\\

Teste: Processos de execução controlada do software

	Visa determinar:

		Se atingiu as especificações.
		Se funciona corretamente pro seu contexto.
		
		Testes visam revelar as falhas do produto.

			| falha --> | erro --> | defeito | <-- erro | <-- falha |

		Depuração é algo não previsto em teste.

			Revelada a presença do falha os defeitos devem ser identificados e corrigidos.

		Passar em todos os testes não significa que o software não tenha defeitos, mas sim significar que os testes não eram bons nem abrangentes.

			Testes só podem mostrar a PRESENÇA de erros, não sua AUSÊNCIA. -Dijkstra-

		Testes destrutivos versus Testes construtivos

			Visão tradicional
			Visão ágil

		Testes se baseiam quase em sua totalidade nos requisitos e seus documentos. Testam o software em todos os estágios do desenvolvimento. E mesmo as menores mudanças nos requisitos podem afetar completamente a abordagem dos testes e seus resultados.

			-Devem ser rastreáveis aos requisitos.
			-Devem ser completamente planejados antes do início.
			-O princípio 80/20 de Paretto se aplica aos testes.
			-Os testes devem inicialmente serem pequenos e depois escalar.

	Conceitos básicos:

		-Casos de teste: descrição de condição particular a ser testada.
		-Procedimento de teste: descrição dos passos necessários para a execução de um ou mais grupos de casos de teste.
		-Critérios de cobertura de teste: percentual de elementos testados.
		-Bateria de testes: um ciclo de execução de todos os procedimentos de teste para uma versão do produto.

	Processos de testes:

		Planejamento de projeto de testes.

			- Planejar testes.

				– Definir contexto e escopo dos testes
				– Estimar esforço
				– Alocar recursos humanos

		Captura de requisitos.

			- Planejar testes.

				– Definir contexto e escopo dos testes
				– Estimar esforço
				– Alocar recursos humanos

		Análise e projeto.

			- Projetar testes.

				– Preparação do ambiente
				– Criar estratégias e diretrizes
				– Identificar casos e procedimentos de teste
				– Especificar critério de aprovação

		Implementação

			- Implementar testes.
			- Executar testes.
			- Avaliar testes.

		As atividades do planejamento:

			- Especificar casos de teste.

				-Especificar entradas e saídas
				-Especificar restrições de uso.
				-Definir dependências entre casos.

			- Definir procedimentos de teste.

				-Definir Requisitos
				-Descrever passos.

		As atividades de execução:

			- Executar testes.

				-Configurar ambiente de teste.
				-Executar procedimentos de teste.
				-Registrar incidentes.

			- Analisar resultados.

				-Registrar dados dos testes.

Classificação SWEBOK.

Classificação DI LUCCA e FASOLINO.

Classificação CRESPO ET AL.

Trabalho prático 5:

	-Definir testes para o projeto inicial. É preciso justificar:
		
		Níveis de teste.

			Teste de unidade.
			Teste de integração.
			Teste de sistema.
			Teste de aceitação.
			Teste de regressão.

		Tipos de teste.

			Teste de funcionalidades.
			Teste de interface.
			Teste de desempenho.
			Teste de carga (stress).
			Teste de usabilidade.
			Teste de volume.

		Técnicas de teste.
		
			Teste funcional.
			Teste estrutural.

	-Pesquise alguma ferramenta de apoio aos testes e apresente por meio dos seguintes itens:
		
		Descrição geral.
		Quais tipos de testes apoia.
		Principais funcionalidades.
		Como os resultados são apresentados.
		
		Lista das ferramentas:
		
			-Selenium.
			-Appium.
			-Cucumber.
			-JMeter.
			-TestNG.
			-WatiR.
			-Cypress.
			-JUnit.
			-TestComplete.
			-Telerik Test Studio.
			-FitNesse.
			
-=-=-=-=-=-=-=-=-=-=-=-\\
| Planejamento de testes    |\\
-=-=-=-=-=-=-=-=-=-=-=-\\
Requisitos

	Página de informações
	Tela de Login
	cadastro
		-Criação
		-Edição
		-Cancelamento
	API de otimizações
	Dashboard com estatísticas
	Dashboard do status da cidade
	Recarga do bilhete integração
	Online 24 horas
	Banco de dados da prefeitura
	Segurança e privacidade
	Efetivar apenas cadastros elegíveis
	Primeira recarga mínima
	Período de carência para cancelamento
	Pagamento
		-QR code
		-Aproximação
	Bloquear motorista em período de penalidades
	Bloquear usuários por fraudes
	Cadastrar motoristas de aplicativos
	Identificar o motorista
	Preservar privacidade dos usuários
	Informar status do tráfego em tempo real
	Informara estatísticas da cidade
	Informar condições de frota
	Sinalizar estados de emergência
	Permitir acesso às instituições de saúde
	Permitir acesso às instituições de segurança pública
	Utilizar servidores e base de dados da CET Rio e
FETRANSPOR
	Informar situações legais das empresas de transporte

Casos de uso

	Cadastrar usuários
	Cancelar cadastro de usuário.
	Otimizar percurso.
	Verificar eventos.
	Recarregar saldo.
	Pagar viagem.
	Cadastrar motorista.
	Cancelar cadastro de motorista.
	Cobrar viagem.
	Validar documentação.
	Sinalizar eventos.
	Otimizar uso de infraestrutura

	Volto em 5

\end{document}