\documentclass[12pt,a4paper]{article} %tipo de documento papel e tamanho da fonte
\usepackage[utf8x]{inputenc} %acentuação
\usepackage{ucs}
\usepackage[portuguese]{babel}
\usepackage[T1]{fontenc} %ajusta textos copiados/colados
\usepackage{amsmath} %símbolos matemáticos
\usepackage{amsfonts} %símbolos matemáticos
\usepackage{amssymb} %símbolos matemáticos
\usepackage{amsthm} %símbolos matemáticos
\usepackage{mathtools} %símbolos matemáticos
\usepackage{dsfont} %símbolos matemáticos
\usepackage{float}
\usepackage{makeidx}
\usepackage{graphicx} %permite inserir figuras
\usepackage{lmodern}
\usepackage{fourier}
\usepackage[left=2cm,right=2cm,top=2cm,bottom=2cm]{geometry} %layout da página
\usepackage{textcomp}
\usepackage{tabto} %permitir tabulação
\pagestyle{empty} %turn off page numbers (currently not working in my TeX distro)
\parindent 0px %turn off indentation
\author{Traian Matisi}
\title{Grafos e Algoritmos}
\begin{document}
\maketitle
\section{Vértices e arestas}
\begin{itemize}
\item As áreas da ponte de Konisberg são pontos ou \textbf{vértices}.
\item As pontes de Konisberg são as arestas ou \textbf{elos}.
\item Dois pontos podem ser ligados por múltiplas arestas, uma aresta, ou mesmo nenhuma aresta.
\end{itemize}
\subsection{Tipos de grafos}
\begin{itemize}
\item Grafos simples
\item Multigrafos
\item Loop
\end{itemize}
No ano 3000 será possível viajar entre os seguintes planetas
\begin{itemize}
\item Terra-Mercúrio
\item Plutão-Vênus
\item Terra-Plutão
\item Plutão-Mercúrio
\item Mercúrio-Vênus
\item Urano-Netuno
\item Netuno-Saturno
\item Saturno-Jupiter
\item Jupiter-Marte
\item Marte-Urano
\end{itemize}
É possível, a partir da Terra, chegar à marte?

\end{document}