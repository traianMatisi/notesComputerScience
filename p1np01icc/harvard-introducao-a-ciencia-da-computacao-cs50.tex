\documentclass[12pt,a4paper]{article} %tipo de documento papel e tamanho da fonte
\usepackage[utf8x]{inputenc} %acentuação
%\usepackage{fontspec} % for use in combo with \setmainfont{}
%\setmainfont{file-name.ttf}[Path=file-path, BoldFont={file-name}, ItalicFont={file-name}, BoldItalicFont={file-name}]
%\setsansfont{file-name.otf}[Path=file-path, BoldFont={...]
\usepackage{ucs}
\usepackage[portuguese]{babel}
\usepackage[T1]{fontenc} %ajusta textos copiados/colados
\usepackage{amsmath} %símbolos matemáticos
\usepackage{amsfonts} %símbolos matemáticos
\usepackage{amssymb} %símbolos matemáticos
\usepackage{amsthm} %símbolos matemáticos
\usepackage{mathtools} %símbolos matemáticos
\usepackage{dsfont} %símbolos matemáticos
\usepackage{float}
\usepackage{makeidx}
\usepackage{graphicx} %permite inserir figuras
\usepackage{lmodern}
\usepackage{fourier}
\usepackage[left=1cm,right=1cm,top=1cm,bottom=1cm]{geometry} %layout da página
%\usepackage{textcomp}
\usepackage{tabto} %permitir tabulação
%\usepackage{hyperref} % You can use \url with \usepackage{hyperref} \url{http://stackoverflow.com/}
%\usepackage{todonotes} %enables you to insert small notes, like \todo{Rewrite this answer \ldots}
%\usepackage{indentfirst}
%\setlength{parindent}{1.1cm}
\pagestyle{empty} %turn off page numbers (currently not working in my TeX distro)
%\parindent 0px %turn off indentation
%\setmainfont{<font name>}
\author{Traian Matisi}
\title{CS50 - Introdução à ciência da computação de Harvard}
%date{}

\begin{document}

\maketitle

\section{Aula 00 - Scratch}
Visualizando algoritmos

O algoritmos é o fundamento da ciência da computação. Algoritmos só existem com a função de resolver problemas. Pra ser um programador, você precisa desenvolver a sua capacidade de resolver problemas e transformá-lo em um algoritmo. $$x_{(problem)}\rightarrow f(x)_{(algorithm)}\rightarrow y_{(solution)}$$

\subsection{Representação de dados}
Computadores usam apenas números, e ainda por cima, apenas números binários, então se quisermos representar qualquer número decimal teremos que converter de binário para decimal. Para letras, teremos que associar um número binário para cada letra (e símbolo), e o mesmo será feito para cores, sons, e qualquer coisa que queiramos representar, usaremos os números inteiros relativos à $2^n$. O padrão precisa ser aceito e oficializado visando faciitar a comunicação entre computadores e comunicação entre profisionais. Os padrôes diferentes como unicode, podemos representar acentos, ideogramas e emojis.

\subsection{Bases numéricas}
\begin{itemize}
\item Binário (0b)\\ 0 e 1, On e Off, High e Low. 
\item Decimal (0d)\\ De zero à dez.
\item Hexadecimal (0x)\\0 1 2 3 4 5 6 7 8 9 A B C D E F
\end{itemize}

\subsection{Ordem O(n)}
Tamanho do problema versus Tempo de resolução
\begin{itemize}
\item O($1$)
\item O($n$)
\item O($n^2$)
\item O($log_{10}n$)
\end{itemize}

\subsection{Pseudocódigo}
\begin{itemize}
\item Adaptação de Linguagem natural para emular uma linguagem de programação.
\end{itemize}

\subsection{Interface gráfica para programação}
\begin{itemize}
\item Blockly
\item Scratch
\item Grasshopper
\end{itemize}

\section{Aula 01 - Programação em C}
Linguagem de programação estruturada

\subsection{helloWorld.c}
\begin{enumerate}
\item \# include <stdio.h>
\item int main(void)\{
\item \tabto{1.1cm}printf("Hello, world".);
\item \}
\end{enumerate}

\subsection{Compilando um programa C}
\begin{itemize}
\item gcc nomeDoPrograma.c -o nomeDoPrograma
\item make nomeDoPrograma
\item clang -args
\begin{itemize}
\item o $\longrightarrow$ OUTPUT
\item l... $\longrightarrow$ LINK COM BIBLIOTECA
\item W...

\end{itemize}
\end{itemize}

\subsection{Anatomia de um programa C}
\begin{itemize}
\item Pré-processamento: procura linhas com \#, compara com os comandos do códiog e importa as bibliotecas necessárias. É seguida de compiling, assembling e linking
\item Bibliotecas: São arquivos .h ou binários que são ligados ao código.
\item Tipos
\item Constantes
\item Variáveis
\item Formatadores
\item Caracteres de escape
\item Argumentos
\item Funções
\item Efeitos colaterais
\item Valores de retorno
\item Blocos
\item Protótipos
\end{itemize}

\subsection{Propriedades de um algoritmo}
\begin{itemize}
\item Correctness\\Resolve todos os problemas a que se propõe?\\Roda e funciona, mas resolve realmente o problema para o qual foi imaginado?
\item Design\\Usa as ferramentas apropriadas?\\Bons argumentos podem ser mal-articulados, bons programas são bem desenhados.
\item Bug free\\Tem algum defeito de lógica ou sintático? 
\item Style\\É legível?\\Segue boas práticas e ou padrões propostos e convenções?
\item Efficiency\\Das ferramentas apropriadas, usa a melhor possível?
\end{itemize}

\subsection{Debuggin}
\begin{itemize}
\item Leia documentação da linguagem.
\item Leia os comentários //(que também são documentação).
\item Use printf/consoleLog.
\item Use as funções das IDEs debug, como breakpoint.
\item Abuse de rubber ducks.
\end{itemize}

\subsection{Comandos comuns no terminal}
\begin{itemize}
\item ls: LIST, mostra arquivos e diretórios, [-a: all], [-l: list]
\item mv: MOVE, move ou renomeia arquivos e diretórios
\item rm: remove/delete file/empty directory [-r: remove non-empty directories]
\item rmdir: remove directory
\item mkdir: make directory/create folder
\item cd: change directory [. means this directory] [.. means higher directory]
\item cp: copy file/directory
\end{itemize}

\subsection{Data types and its formaters}
Algumas vezes, alguns numeros são grandes demais ou "decimal" demais e precisamos usar os tipos apropriados.
\begin{itemize}
\item int: (\%i ou \%d) [32 bits ou 4 bytes] conta de até
\item float: (\%f) [32 bits ou 4 bytes] conta de até 
\item double: (\%lf) [64 bits ou 8 bytes] conta de até
\item bool: (\%b) [8 bits ou 1 byte] true or false
\item char:  (\%c) [8 bits ou 1 byte] conta de até
\item string: (\%c) [not real in C, it's actually an array]
\item long: (\%li) [64 bits ou 8 bytes] conta de até
\item short: (\%i ou \%d) [16 bits ou 2 bytes] conta de até
\item signed: (\%i ou \%d) [32 bits ou 4 bytes] conta de até
\item unsigned: (\%i ou \%d) [bits ou bytes] conta de até
\end{itemize}
Some operations, like division, needs that the terms be both the same type, as to not lose data in decimal values. In the case we dont have acces to the type, we can cast the type over the old type

\subsection{Mathematical operators}
Um erro comum é a falta de casting. Exemplo, numa divisão entre dois inteiros com resto, sabemos que devemos guardar o resultado num float ou double. Mas o resultado de dois inteiros é outro inteiro, então antes devemos "type cast" o float/double nos operandos que eram inteiros pra conseguir o resultado em float. É possível também usar apenas um número com vírgula pra forçar o compilador a usar o \textit{casting} implicitamente.
\begin{itemize}
\item soma
\item diferença
\item multiplicação
\item divisião
\item resto (mod)
\item atribuições seguidas de operações\\+=; -=; *=; /=; \%=.\\++; --.
\end{itemize}

\subsection{Logical operators}
\begin{itemize}
\item igualidade
\item desigualdade
\item maior que
\item maior ou igual
\item menor que
\item menor ou igual
\item não
\item e
\item ou inclusivo
\item ou exclusivo
\end{itemize}

\subsection{Condicionais}
Se, senão.
\begin{enumerate}
\item \# include <stdio.h>
\item int main(void)\{
\item \tabto{1.1cm}if (x < y)\{
\item \tabto{2.2cm}printf("x é maior do que y");
\item \tabto{1.1cm}\}
\item \tabto{1.1cm}else\{
\item \tabto{2.2cm}printf("x não é maior do que y");
\item \tabto{1.1cm}\}
\item \}
\end{enumerate}
Se, senão se, senão.
\begin{enumerate}
\item \# include <stdio.h>
\item int main(void)\{
\item \tabto{1.1cm}if (x == y)\{
\item \tabto{2.2cm}printf("x é igual a y");
\item \tabto{1.1cm}\}
\item \tabto{1.1cm}else if (x < y)\{
\item \tabto{2.2cm}printf("x é menor do que y");
\item \tabto{1.1cm}\}
\item \tabto{1.1cm}else\{
\item \tabto{2.2cm}printf("x é maior do que y");
\item \tabto{1.1cm}\}
\item \}
\end{enumerate}

\subsection{Laços}
\textbf{Laço "enquanto" infinito}
\begin{enumerate}
\item \# include <stdio.h>
\item int main(void)\{
\item \tabto{1.1cm}while (true)\{
\item \tabto{2.2cm}printf("Laço infinito");
\item \tabto{1.1cm}\}
\item \}
\end{enumerate}
\textbf{Laço "enquanto" com contador (condicional)}
\begin{enumerate}
\item \# include <stdio.h>
\item int main(void)\{
\item \tabto{1.1cm}int i = 0;
\item \tabto{1.1cm}int n = 100;
\item \tabto{1.1cm}while (i < n)\{
\item \tabto{2.2cm}printf("contador = \%i", i);
\item \tabto{2.2cm}i += 1; //ou também i++
\item \tabto{1.1cm}\}
\item \}
\end{enumerate}
\textbf{Laço "enquanto, faça" com contador (condicional)}
\begin{enumerate}
\item \# include <stdio.h>
\item int main(void)\{
\item \tabto{1.1cm}int i = 0;
\item \tabto{1.1cm}int n = 100;
\item \tabto{1.1cm}do\{
\item \tabto{2.2cm}printf("contador = \%i", i);
\item \tabto{2.2cm}i += 1; //ou também i++
\item \tabto{1.1cm}\}
\item \tabto{1.1cm}while (i < n);
\item \}
\end{enumerate}
\textbf{Laço "faça"}
\begin{enumerate}
\item \# include <stdio.h>
\item int main(void)\{
\item \tabto{1.1cm}for (int i = 0; i < 100; i++)\{
\item \tabto{2.2cm}printf("contador = \%i", i);
\item \tabto{1.1cm}\}
\item \}
\end{enumerate}

\subsection{Funções}
É necessário declarar no pré processamento, e definir após o \textit{main}.\\
Ao definir e ao declarar (sempre nos dois) podemos pedir argumentos.\\
Sempre tomar cuidado com escopo nas funções e nas estruturas de controle.\\
Podemos passar qualquer mudança dentro da função (retorno) por referência (ponteiros).\\
\subsection{Recursividade}
Rec

\section{Aula 02 - Vetores}
\subsection{array.c}
Índices começam a contar de 0. Podemos colocar variáveis para definir ou alcançar índices
\begin{enumerate}
\item \# include <stdio.h>
\item int main(void)\{
\item \tabto{1.1cm}int numerosPrimos[] = \{2, 3, 5, 7, 11, 13, 17, 19, 23\};
\item \tabto{1.1cm}int numerosFibonacci[9] = \{0, 1, 1, 2, 3, 5, 8, 13, 21\};
\item \}
\end{enumerate}

\section{Aula 03 - Algoritmos}

\section{Aula 04 - Memória}

\section{Aula 05 - Estrutura de dados}

\section{Aula 06 - Python}

\section{Aula 07 - SQL}

\section{Aula 08 - HTML; CSS; JavaScript}

\section{Aula 09 - Flask}

\section{Aula 10 - Ética}

\section{Aula 11 - Cyber Segurança}

\section{Aula 12 - Inteligência artificial}

\section{Aula 13 - Emoji}

\end{document}