\documentclass[12pt,a4paper]{article} %tipo de documento papel e tamanho da fonte

\usepackage[utf8x]{inputenc} %acentuação
\usepackage{ucs}
\usepackage[portuguese]{babel}
\usepackage[T1]{fontenc} %ajusta textos copiados/colados
\usepackage{amsmath} %símbolos matemáticos
\usepackage{amsfonts} %símbolos matemáticos
\usepackage{amssymb} %símbolos matemáticos
\usepackage{amsthm} %símbolos matemáticos
\usepackage{mathtools} %símbolos matemáticos
\usepackage{dsfont} %símbolos matemáticos
\usepackage{float}
\usepackage{makeidx}
\usepackage{graphicx} %permite inserir figuras
\usepackage{lmodern}
\usepackage{fourier}
\usepackage[left=2cm,right=2cm,top=2cm,bottom=2cm]{geometry} %layout da página
\usepackage{textcomp}
\usepackage{tabto} %permitir tabulação
\usepackage{hyperref} % You can use \url with \usepackage{hyperref} \url{http://stackoverflow.com/}
\usepackage{todonotes} %enables you to insert small notes, like \todo{Rewrite this answer \ldots}

\pagestyle{empty} %turn off page numbers (currently not working in my TeX distro)
\parindent 0px %turn off indentation

\title{Notas de aula - Programação estruturada 2021-1}
\author{Traian Matisi}

\begin{document}
\maketitle

\section{Algoritmos}

\subsection{Linguagem natural}
\begin{itemize}
\item Linguagem natural:
Imensa capacidade de comunicar.
\item Linguagem algoritmica:
Limitada e restrita visando precisão e objetividade, procura evitar ambguidade.
\item Pseudocódigo:
Usa linguagem natural de maneira algoritmica visando um meio termo, criando assim uma ponte entre o mundo real e o progrma que se deseja.
\item Fluxograma:
Representação ilustrada do pseudocódigo, visa orientar melhor o programador, cada tipo de algoritmo é representado por uma forma geométrica.
\item Algoritmo:
Sequência de passos bem definidos, sem ambiguidade, finito e que se propõe a solucionar algum problema. Se assemelha a uma receita. Os algoritmos possuem estruturas básicas que emulam as propriedades das linguagens de programação. Os problemas sempre possuem mais de uma solução, e portanto existem mais de um algoritmos para um mesmo problema (mas qual é a definição de problema?).
\item Problema:
É um conjunto de dados de entrada (também chamado de instância) e uma questão formulada sobre essa instância.
\item As características fundamentais dos algoritmos
\begin{itemize}
\item Terminar em tempo finito.
\item Passos definidos e sem ambiguidade.
\item Deve conter zero ou mais entradas.
\item Deve fornecer alguma saída.
\item Deve funcionar
\item Exemplo $\rightarrow ax+b=0 \rightarrow x=-b/a$\\	"a" e "b" são as entradas e x é a saída.
\end{itemize}
\item Escrevendo o algoritmo para calcular x:
\begin{enumerate}
\item leia a ($a \leftarrow$ primeira entrada)
\item leia b ($b \leftarrow$ segunda entrada)
\item $x \leftarrow (-b/a)$
\item escreva $x$
\end{enumerate}
\item Já temos aqui os conceitos de variáveis e operadores aritméticos. Perceba que matematicamente o algoritmo acima não trata da entrada de a = 0. Pra isso teríamos que verificar antes se há condição de realizar o código, e para isso exitem estruturas básicas em programação pra tratar possíveis erros.
\item Pseudocódigo, achar X numa equação de primeiro grau:
\begin{enumerate}
\item Ler a
\item Ler b
\item Se $(a=0)$ então
\item \tabto{1.1cm}Escreva: "A equação não tem solução $\mathbb{R}$."
\item Senão
\item \tabto{1.1cm}$x \leftarrow (\frac{-b}{a})$
\item Fim-Se
\end{enumerate}
\item Pseudocódigo, calcular média das notas:
\begin{enumerate}
\item Ler nota $a$
\item Ler nota $b$
\item $x \leftarrow \frac{a+b}{2}$
\item Escrever $x$
\end{enumerate}
É possivel criar outro fluxo ou mesmo estender esse para verificar se a nota é suficiente para passar
\item Pseudocódigos precisam ser simples e objetivos:\\
1 Um verbo por frase\\
2 Precisa ser compreensível por qualquer público\\
3 Frases curtas e simples\\
4 Ser objetivo\\
5 Não ser ambíguo\\
6 Palavras em negrito significam palavras reservadas para instruções\\
7 Palavras em itálico significam variáveis\\
8 No papel escrito não há necessiddade de negrito ou itálico
\item Símbolos de fluxograma:\\
1 Cilindro: Símbolo para início e fim, podem existir mais de um símbolo final\\
2 Retângulo: Símbolo de processamento\\
3 Losango: Símbolo de tomada de decisões\\
4 Trapezóide: Símbolo de entrada de dados\\
5 Skate pipe: Símbolo de saída de dados\\
6 Pequeno círculo: Conector de fluxos entre páginas ou entre ramos\\
\item Como exercício, realizar os dois pseudocódigos acima com os seis símbolos aprendidos.
\item Existem mais símbolos além desses seis para concorrência, verificação de erros, etc. serão abordados num futuro próximo.
\item Tipos de comandos:\\
1 Ler\\
2 Escrever\\
3 Atribuição\\
4 Dados booleanos\\
5 Dados numéricos\\
6 Dados alfanuméricos\\
7 Controle: se, senão\\
8 Controle: caso\\
n Repetição: enquanto\\
n Repetição: para\\
n Repetição: faça... enquanto\\
n Procedimentos:\\
\end{itemize}

\section{Conteúdo online}

\subsection{Sites}
\begin{enumerate}
\item BLOCKLY: < https://blockly.games/ >
\item blockly: < https://developers.google.com/blockly/ >
\item codepen: < https://codepen.io/ >
\item codepad: < https://codepad.co/ >
\item playcode: < https://playcode.io/ >
\item JsFiddle: < https://jsfiddle.net/ >
\item CSS deck: < https://cssdeck.com/ >
\item beecrowd: < https://www.beecrowd.com.br/ >
\item exercism: < https://exercism.org/ >
\item codewars: < https://www.codewars.com/ >
\item hackerRank: < https://www.hackerrank.com/ >
\item leetCode: < https://leetcode.com/ >
\item stack overflow: < https://stackoverflow.com/ >
\item stack exchange: < https://tex.stackexchange.com/ >
\item Toptal: < https://www.toptal.com/web\# contract-just-incredible-coders-now >
\item webdeveloper.com: < https://www.webdeveloper.com/ >
\item Github: < https://github.com/ >
\item Mozilla developers: < https://developer.mozilla.org/ >
\item SAP: < https://community.sap.com/ >
\item Expert Exchange: < https://www.experts-exchange.com/ >
\item r/webdev, fórum do Reddit: < https://www.reddit.com/r/webdev/ >
\item Bootstrap, grupo do Slack: < https://bootstrap-slack.herokuapp.com/ >
\item DEV: < https://dev.to/ >
\item coderwall: < https://coderwall.com/ >
\item Designer Hangout: < https://www.designerhangout.co/ >
\item Bytes: < https://bytes.com/ >
\item coffeCup: < https://www.coffeecup.com/forums/ >
\item hashNode: < https://hashnode.com/ >
\item BitDegree: < https://www.bitdegree.org/learn/ >
\item Coursera: < https://www.coursera.org/ >
\item Udemy: < https://www.udemy.com/ >
\item code.org: < https://code.org/ >
\item codeAcademy: < https://www.codecademy.com/ >
\item freeCodeCamp: < https://www.freecodecamp.org/ >
\item codeConquest: < https://www.codeconquest.com/ >
\item odinProject: < https://www.theodinproject.com/ >
\item edX: < https://www.edx.org/ >
\item MIT open courseware: < https://www.codeconquest.com/ >
\item Khan Academy: < https://www.khanacademy.org/ >
\item Brilliant: < https://brilliant.org/ >
\item Dash General Assembly: < https://dash.generalassemb.ly/ >
\item w3schools: < https://www.w3schools.com/ >
\item hackr.io: < http://hackr.io/ >
\item bento: < https://bento.io/ >
\item code avengers: < https://www.codeavengers.com/ >
\item Solo Learn: < https://www.sololearn.com/ >
\item Google Developers Training: < https://developers.google.com/training >
\item Developers.Android.com: < https://developer.android.com/index.html >
\item Developers.Google.com: < https://developers.google.com/ >
\item Upskill: < https://upskillcourses.com/ >
\item Plural Sight: < http://pluralsight.com/ >
\item codeasy.net: < https://codeasy.net/ >
\item hack.pledge(): < https://hackpledge.org/ >
\item aGupie Ware: < http://blog.agupieware.com/ >
\item edabit: < http://edabit.com/ >
\item instructables: < https://developers.google.com/ >
\item Code Combat: < https://br.codecombat.com/ >
\item CodaKid
\item Byjuu
\item CodeSpark Academy
\item Create \& Learn
\item Bit Degree
\item Tynker
\item Coding With Kids
\item Codemoji
\item linkedIN learning - 600 reais anuais
\item Lynda: < https://www.lynda.com/ >
\item general assembly dash - Web development
\item upskill
\item MDN web docs
\item edX (CS50)
\item https://pll.harvard.edu/catalog/free
\item https://www.algoexpert.io - 62 dólares por ano
\end{enumerate}

\subsection{Aplicativos}
\begin{enumerate}
\item Scratch
\item Bee-bot
\item Grasshoper
\item Evernote
\item Notion
\item Obsidian
\item QPython
\end{enumerate}

\section{Operadores}

\subsection{Operadores aritméticos}
\begin{itemize}
\item Soma ($a + b = c$)
\item Subtração ($a - b = c$)
\item Multiplicação ($a \times b = c$) ou ($a \cdot b = c$) ou ($a * b = c$)
\item Divisão ($a \div b = c$) ou ($\frac{a}{b} = c$) ou ($a/b = c$)
\item Resto da divisão ($a _{mod}b = c$) ou ($a \% b = c$)
\end{itemize}

\subsubsection{Operadores relacionais}
\begin{itemize}
\item Igual $\rightarrow$ =
\item Maior $\rightarrow$ >
\item Maior ou igual $\rightarrow$ >=
\item Menor $\rightarrow$ <
\item Menor ou igual $\rightarrow$ <=
\item Diferente $\rightarrow$ !=
\end{itemize}

\subsection{Operadores lógicos}
\begin{itemize}
\item E $\rightarrow$ \&\&
\item Ou $\rightarrow$ ||
\item Não $\rightarrow$ !
\end{itemize}

\subsection{Exercícios}
Escreva uma pseudocódigo para cada item abaixo:
\begin{enumerate}
\item Somar 3 números
\item Calcular a média de um aluno onde:\\Média = (2 * Prova1 + 2 * Prova2 + Grupo)/5\\Grupo = (Participação + 2 * Trabalho)/3
\item Ler o nome de duas pessoas, trocar esses nomes nas variáveis e imprimí-los na tela
\item Calcular a média das provas de uma turma de 60 alunos
\item Determinar a maior nota entre as provas dos 60 alunos
\item Determinar quantaas vezes essa nota ocorreu
\item Em posse de 3 vasilhas de 4L, 9L e uma maior sem medida, encher a maior com 6L
\item Em posse de 3 vasilhas de 4L, 9L e uma maior sem medida, encher a maior com 35L
\item De posse de $3^{n}$ moedas de mesmo pesso exceto por uma e uma balança de precisão, elabora um algoritmo que permita identificar rapidamente qual a moeda
\item Um algoritmo que informe o número de passos para a conjectura de Ulam para os números inteiros x maior que 1 e menor que 10\\Para tal, considere:\\Se x for ímpar, x = 3x + 1\\Se x for par, x = x/2\\Repetir até que x = 1
\item Aproximação de $\pi$
\end{enumerate}
Tente fazer no VisuAlg ou Portugol estúdio, ambos em conjunto com fluxogramas.

\end{document}